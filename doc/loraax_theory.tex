\documentclass[11pt]{article}

\usepackage{amssymb}
\usepackage{amsmath}
\usepackage{setspace}
\usepackage{graphicx}
\usepackage{subfig}
\usepackage{color}
\usepackage[left=1.2in, right=1in, top=1in, bottom=1in]{geometry}
\usepackage{verbatim}
\usepackage{hyperref}

\begin{document}

\title{LORAAX Theory Guide}
\date{\today}
\maketitle

\tableofcontents

\section{Boundary Conditions}\label{sec:bcs}

The Neumann-type boundary condition specifies 0 normal flow through a panel:

\begin{equation}
\mathbf{V}\cdot{\mathbf{\hat{n}}} = 0,
\label{eq:bc1}
\end{equation}

\noindent where $\mathbf{\hat{n}}$ is the outward-facing normal vector at the centroid of
the panel. The velocity is the sum of contributions from all source and doublet panels on
wing surfaces, doublet panels in the wake, plus the freestream influence, as follows:

\begin{equation}
\mathbf{V}(x,y,z) = \mathbf{V}_\infty
                  + \sum_{i=1}^{\text{pan}}\left(\mathbf{a}_i(x,y,z)\mu_i
                  +                              \mathbf{b}_i(x,y,z)\sigma_i\right)
                  + \sum_{i=\text{pan}+1}^{\text{wakepan}}\mathbf{a}_i(x,y,z)\mu_i,
\label{eq:vel1}
\end{equation}

\noindent where $\mathbf{a}_i$ and $\mathbf{b}_i$ are the influence
coefficients due to the
doublet panel $i$ and source panel $i$, respectively, and $\mu$ and $\sigma$ are
the strengths of these elements, assumed constant on a panel.

Equation \ref{eq:vel1} gives the boundary condition in the Neumann form. For the purpose
of evaluating it numerically, it is sometimes preferred to represent the boundary
condition in Dirichlet form. In this case, it is recognized that Eq. \ref{eq:vel1} is
equivalent to the velocity potential, $\Phi$, being constant inside the body. $\Phi$ is
related to the velocity as follows:

\begin{equation}
\Phi = \nabla\mathbf{V}.
\label{eq:phi1}
\end{equation}

For the source-doublet panel formulation,

\begin{equation}
\Phi(x,y,z) = \Phi_\infty
            + \sum_{i=1}^{\text{pan}}\left[c_i(x,y,z)\mu_i + d_i(x,y,z)\sigma_i\right]
            + \sum_{i=\text{pan}+1}^{\text{wakepan}+\text{pan}}c_i(x,y,z)\mu_i,
\label{eq:phi2}
\end{equation}

\noindent where $c_i$ and $d_i$ are the influence coefficients for the velocity potential
of doublet and source panels $i$, respectively acting on the point $(x,y,z)$. Note that in
this formulation, the influence coefficients are scalars instead of vectors as in Eq.
\ref{eq:vel1}. The boundary
condition states that $\Phi$ is constant inside the body, so

\begin{equation}
   \Phi_\infty
 + \sum_{i=1}^{\text{pan}}\left[c_i(x,y,z)\mu_i + d_i(x,y,z)\sigma_i\right]
 + \sum_{i=\text{pan}+1}^{\text{wakepan}+\text{pan}}c_i(x,y,z)\mu_i = \text{const},
\label{eq:phi3}
\end{equation}

\noindent where it is assumed that the point $(x,y,z)$ lies inside the body. Different
values for the constant are possible, but a popular choice is $\text{const} =
\Phi_\infty$, which results in the boundary condition

\begin{equation}
\sum_{i=1}^{\text{pan}}\left[c_i(x,y,z)\mu_i + d_i(x,y,z)\sigma_i\right]
\sum_{i=\text{pan}+1}^{\text{wakepan}+\text{pan}}c_i(x,y,z)\mu_i = 0.
\label{eq:phi4}
\end{equation}

This choice of the internal potential requires the source strengths to be

\begin{equation}
\sigma_i = -\mathbf{V}_\infty\cdot\mathbf{\hat{n}}_i,
\label{eq:sigma1}
\end{equation}

\noindent where $\mathbf{\hat{n}}_i$ is the outward-pointing normal direction at
the centroid of panel $i$ \textcolor{red}{(cite Katz \& Plotkin)}.

Finally, the Kutta condition must be enforced, which states that no jump in $\Phi$ is
allowed at the trailing edge. This is done by setting the doublet strength in the wake
panels along the trailing edge as

\begin{equation}
\mu_{\text{te}} = \mu_{\text{te,top}} - \mu_{\text{te,bot}},
\label{eq:wake_strength}
\end{equation}

\noindent where \emph{te} represents the portion of the wing's trailing edge
spanned by a given column of surface and trailing wake panels.

\section{System of Equations}

Applying Eq. \ref{eq:phi4} at every surface panel, and substituting in Eqs.
\ref{eq:sigma1} and \ref{eq:wake_strength}, Eq. \ref{eq:system1} is obtained.
Here, the identifier $(x,y,z)$ has been removed, as it is implicit that this equation is
to be applied at the centroid of a panel, just inside the body.

\begin{equation}
  \sum_{i=1}^{\text{pan}}c_i\mu_i 
+ \sum_{k=1}^{N_\text{te}}e_k\left(\mu_{k,\text{top}} - \mu_{k,\text{bot}}\right)
= \sum_{i=1}^{\text{pan}}d_i\mathbf{V}_\infty\cdot\mathbf{\hat{n}}_i
\label{eq:system1}
\end{equation}

\noindent Note that the unknowns are the doublet strengths $\mu$ only, while the
known quantities have been moved to the right hand side. The influence
coefficients $e_k$ represent the potential due to the strip of wake doublets
behind the trailing edge spanwise location $k$ with unit strength
Equation \ref{eq:system1} can be
applied to each of the surface panels, with only the influence coefficients
$c$, $e$, and $d$ changing. The result is a system of
$N_{\text{pan}}$ equations with $N_{\text{pan}}$ unknown doublet strengths. It
can be represented in matrix form as follows:

\begin{equation}
\mathbf{\overline{AIC}}\boldsymbol{\mu} = \mathbf{RHS},
\label{eq:system2}
\end{equation}

\noindent where $\mathbf{\overline{AIC}}$ is the matrix of aerodynamic influence
coefficients,
$\boldsymbol{\mu}$ is the vector of unknown doublet strengths, and $\mathbf{RHS}$ is the
vector of known quantities on the right hand side of the equation. $\mathbf{\overline{AIC}}$
is a
dense square matrix, so a direct method like LU decomposition is most appropriate to solve
it.

\end{document}
