\documentclass[11pt]{article}

\usepackage{amssymb}
\usepackage{amsmath}
\usepackage{setspace}
\usepackage{graphicx}
\usepackage{subfig}
\usepackage{color}
\usepackage[left=1.2in, right=1in, top=1in, bottom=1in]{geometry}
\usepackage{verbatim}
\usepackage{hyperref}

\begin{document}

\title{LORAAX Theory Guide}
\date{\today}
\maketitle

\tableofcontents

\section{Boundary Conditions}\label{sec:bcs}

LORAAX uses Neumann-type boundary conditions on wing surfaces, which specify 0
normal flow through a panel:

\begin{equation}
\mathbf{V}\cdot{\mathbf{\hat{n}}} = 0.
\label{eq:bc1}
\end{equation}

The velocity is the sum of all sources, doublets, and vortices, plus the freestream
influence, as follows:

\begin{equation}
\mathbf{V} = \sum_{i=1}^{N_{\text{pan}}}\left(\mathbf{a}_i\mu_i + \mathbf{b}_i\sigma_i\right)
           + \sum_{j=1}^{N_{\text{vort}}}\mathbf{c}_i\gamma_i + \mathbf{V}_\infty,
\label{eq:vel1}
\end{equation}

\noindent where $\mathbf{a}_i$ and $\mathbf{b}_i$ are the influence
coefficients due to the
doublet panel $i$ and source panel $i$, respectively, $\mathbf{c}_j$ is the
influence
coefficient due to wake vortex ring or horseshoe $j$, and $\mu$, $\sigma$, and
$\gamma$ are distributed doublet, source, and circulation strengths of these
elements, respectively.

\section{System of Equations}

Following from the boundary conditions, a system of equations can be formed to
solve for the unknown doublet, source, and circulation strengths. However, the
system is underconstrained if each of these is considered an unknown. First, one
of the source or doublet strengths on each panel must be specified. Typically,
the source strength is specified as

\begin{equation}
\sigma_i = -\mathbf{V}_\infty\cdot\mathbf{\hat{n}}.
\label{eq:sigma1}
\end{equation}

This form takes care of the ``zeroth-order'' freestream portion of the boundary
condition, allowing for smaller doublet strengths. During the first iteration
prior to unsteady wake rollup, the entire line of wake vortices behind a set of
top- and bottom-surface trailing edge panels is

\begin{equation}
\gamma_{\text{te}} = \mu_{\text{te,top}} - \mu_{\text{te,bot}},
\label{eq:wake_circ}
\end{equation}

\noindent where \emph{te} represents the portion of the wing's trailing edge
spanned by a given column of panels and wake vortices. This relation is required
by the Kutta condition; it results in no jump in singularity strength across the
trailing edge. Rewriting the equations above, the boundary condition for a given
panel can be represented as

\begin{equation}
  \sum_{i=1}^{N_{\text{pan}}}\mathbf{a}_i\mu_i\cdot\mathbf{\hat{n}}
+ \sum_{k=1}^{N_{\text{te}}}\mathbf{d}_k
    \left(\mu_{k,\text{top}} - \mu_{k,\text{bot}}\right)\cdot\mathbf{\hat{n}}
= -\mathbf{V}_\infty\cdot\mathbf{\hat{n}}
+ \sum_{i=1}^{N_{\text{pan}}}\left(\mathbf{V}_\infty\cdot\mathbf{\hat{n}}_i\right)
    \left(\mathbf{b}_i\cdot\mathbf{\hat{n}}\right).
\label{eq:system1}
\end{equation}

\noindent Note that the unknowns are the doublet strengths $\mu$ only, while the
known quantities have been moved to the right hand side. The influence
coefficients $\mathbf{d}_k$ represent the velocity due to the strip of wake
vortices behind the trailing edge spanwise location $k$ with unit circulation
strength. Equation \ref{eq:system1} can be
applied to each of the surface panels, with only the influence coefficients
$\mathbf{a}$, $\mathbf{b}$, and $\mathbf{d}$ changing. The result is a system of
$N_{\text{pan}}$ equations with $N_{\text{pan}}$ unknown doublet strengths. It
can be represented in matrix form as follows:

\begin{equation}
\mathbf{\overline{AIC}}\boldsymbol{\mu} = \mathbf{RHS},
\label{eq:system2}
\end{equation}

\noindent where $\mathbf{\overline{AIC}}$ is the matrix of aerodynamic influence
coefficients,
$\boldsymbol{\mu}$ is the vector of unknown doublet strengths, and $\mathbf{RHS}$ is the
vector of known quantities on the right hand side of the equation. $\mathbf{\overline{AIC}}$
is a
dense square matrix, so a direct method like LU decomposition is most appropriate to solve
it.

\end{document}
