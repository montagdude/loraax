\documentclass[11pt]{article}

\usepackage{amssymb}
\usepackage{amsmath}
\usepackage{setspace}
\usepackage{graphicx}
\usepackage{subfig}
\usepackage{color}
\usepackage[left=1.2in, right=1in, top=1in, bottom=1in]{geometry}
\usepackage{verbatim}
\usepackage{hyperref}

\begin{document}

\title{LORAAX User Guide}
\date{\today}
\maketitle

\tableofcontents

\section{Input Files}

Performing an analysis in LORAAX requires two input XML files: one specifying
the conditions and settings for the analysis and the other defining the
aircraft geometry.

\subsection{Analysis Input File}

The structure of the analysis input file is as follows:
\begin{verbatim}
<Main>
    <!--
    Main inputs 
    -->
    <XfoilRunOptions>
        <!--
        Xfoil analysis inputs
        -->
    </XfoilRunOptions>
    <XfoilPaneling>
        <!--
        Xfoil paneling inputs
        -->
    </XfoilPaneling>
</Main>
\end{verbatim}

\subsubsection{Main}

\begin{itemize}
	\item CaseName: String. Required: Yes. Description: Name identifying the
		case. It serves as the prefix for various output files.
	\item GeometryFile: String. Required: Yes. Description: Name of the geometry
		XML file for the case.
	\item FreestreamSpeed: Float. Required: Yes. Description: Magnitude of the
		freestream velocity vector. See note on units below.
	\item FreestreamStaticPressure: Float. Required: Yes. Description: Static
		pressure of freestream flow. See note on units below.
	\item FreestreamDensity: Float. Required: Yes. Description: Density of
		freestream flow. See note on units below.
	\item FreestreamViscosity: Float. Required: Yes. Description: Dynamic
		(absolute) viscosity of freestream flow. See note on units below.
	\item AngleOfAttack: Float. Required: Yes. Description: Angle of attack of
		freestream flow, in degrees. If 0, the freestream flow is in the
		x-direction; if 90, the freestream flow is in the z-direction, etc.
	\item RollupWake: Boolean. Required: No. Default: false. Description: Whether to deform
		the wake into its correct shape as part of the solution. If false, the
		wake will remain planar and aligned with the freestream.
	\item RollupDist: Float. Required: Only if RollupWake is true. Description:
		Length of the deforming part of the wake. The remaining wake behind this
		distance extending to infinity is assumed to be planar and aligned with
		the freestream.
	\item WakeIters: Integer. Required: Only if RollupWake is true. Description:
		Number of iterations for a particle moving at the freestream speed to
		traverse RollupDist. Dividing RollupDist by WakeIters gives the
		approximate length of a wake panel.
	\item InitialWakeAngle: Float. Required: No. Default: The angle of attack.
		Description: In some cases, it may be desirable to change the initial wake angle and
		allow the roll-up process to move it to its correct position; for
		example, if the initial planar wake would intersect a downstream
		stabilizer, or to improve stability of the solution. This setting only
		takes effect if RollupWake is true. If enabled, MinIters should be no
		less than WakeIters.
	\item Viscous: Boolean. Required: Yes. Description: Whether to perform a
		viscous or inviscid analysis.
	\item StoppingTolerance: Float. Required: No. Default: $1\times 10^{-5}$.
		Description: If the lift coefficient changes by less than this value
		from one iteration to the next, the solution is considered converged.
	\item MaxIters: Integer. Required: No. Default: 100. The maximum number of
		iterations to perform. The analysis will stop after this number even
		if the lift convergence criterion has not been met.
	\item MinIters: Integer. Required: No. Default: 0. Description: The minimum number of
		iterations to perform. The analysis will continue for at least this
		number of iterations even if the lift convergence criterion has been
		met.
	\item VisualizationFrequency: Integer. Required: No. Default: 1.
		Description: How often to write visualization files (surface+wake and
		sectional data), in number of iterations. If 0, these files will be
		written only at the end of the analysis.
\end{itemize}

\subsubsection{XfoilRunOptions}

\begin{itemize}
	\item ncrit: Float. Required: No. Default: 9. Description: Critical
		log amplification ratio of boundary layer disturbances for
		laminar-turbulent transition.
	\item xtript: Float. Required: No. Default: 1. Description: Boundary layer
		forced transition location on top surface, in terms of $x/c$.
	\item xtripb: Float. Required: No. Default: 1. Description: Boundary layer
		forced transition location on bottom surface, in terms of $x/c$.
	\item maxit: Integer. Required: No. Default: 100. Maximum number of Xfoil
		iterations for each section during each LORAAX analysis iteration.
	\item vaccel: Float. Required: No. Default: 0.01. Description: Xfoil
		viscous solution convergence acceleration parameter. In theory, a value of 0 results
		in the fastest convergence but possibly less stability, but in practice
		very little influence may be seen. It has no effect on the final
		converged viscous solution (if obtained). See Xfoil's documentation for
		more information.
\end{itemize}

\subsubsection{XfoilPaneling}

\begin{itemize}
	\item npan: Integer. Required: No. Default: 9. Description: Number of panels
		around the airfoil for the Xfoil solution. It is independent of the 3D
		wing paneling, which is defined in the aircraft geometry XML file.
	\item cvpar: Float. Required: No. Default: 1. Description: Curvature
		attraction parameter for airfoil paneling. A value of 0 results in
		uniform spacing, and a value of 1 results in strong clustering in
		areas of high curvature.
	\item cterat: Float. Required: No. Default: 0.15. Description: Ratio of
		trailing edge to leading edge panel density.
\end{itemize}

\subsubsection{Note on Units}

LORAAX does not perform any internal unit conversions; it assumes that all
units, including the length units for the aircraft geometry file and the
freestream conditions, are consistent. This means that the units of force are
equal to the units of mass times length divided by seconds squared. In practice,
this means that if, for example, the geometry is defined in meters, then
freestream pressure should be specified in Pascals (N/m$^2$), viscosity should
be in kg/m-s, and the resulting lift and drag will be in Newtons.
Consistent units are also important because no conversions are performed when
computing the Reynolds number:

\begin{equation}
	Re = \frac{\rho U_\infty \overline{c}}{\mu}
	\label{eq:reynolds_number}
\end{equation}

\noindent and the Mach number:

\begin{equation}
	M_\infty = \sqrt{1.4\frac{p_\infty}{\rho_\infty}}.
	\label{eq:mach_number}
\end{equation}

If units are not consistent, then the non-dimensional lift, drag, and pitching
moment may be wrong as well as the dimensional forces and moments. Of course,
if one is only concerned with the non-dimensional coefficients, it is valid to
``design'' the freestream conditions to achieve a desired
Mach and Reynolds number without regard for the actual units. This may be done
by, for instance, setting some of the freestream conditions to a fixed value
and computing the other conditions using the Eqs. \ref{eq:reynolds_number} and
\ref{eq:mach_number}.
As long as those two parameters are correct, then the
non-dimensional coefficients will be correct even though the dimensional forces
and moments may not be meaningful. For convenience, Table \ref{table:units}
lists atmospheric conditions at sea level in some common unit systems.

\begin{table}
\caption{Atmospheric conditions at sea level}
\centering
	\begin{tabular}{c c c} \hline \hline
	Static pressure                  & Density                          & Viscosity \\
	101325 N/m$^2$                   & 1.2252 kg/m$^3$                  & 1.7894$\times$10$^{-5}$ kg/m-s \\
	2116.22 lb$_\text{f}$/ft$^2$ & 2.3773$\times$10$^{-3}$ slug/ft$^3$  & 3.7372$\times$10$^{-7}$ slug/ft-s \\
	14.6959 lb$_\text{f}$/in$^2$ & 1.1465$\times$10$^{-7}$ snail/in$^3$ & 2.5953$\times$10$^{-9}$ snail/in-s
	\end{tabular}
\label{table:units}
\end{table}

\subsection{Geometry Input File}

This file defines the aircraft geometry. An aircraft is made up of a number of
wings. Each wing consists of a number of sections and airfoils. Sections are
connected at the leading and trailing edges by straight lines, but curvature
can be approximated by a number of straight segments. x is the chordwise
direction and y is the spanwise direction. Only the right side of a wing may
be defined; y = 0 is assumed to be a symmetry plane. The structure of the
geometry input file is as follows:
\begin{verbatim}
<Aircraft>
    <Reference>
        <!--
        Reference areas, lengths, and points
        -->
    </Reference>
    <Wing>
        <Paneling>
            <!--
            Paneling inputs
            -->
        </Paneling>
        <Sections>
            <Section>
                <!--
                Section geometry definition
                -->
            </Section>
            <Section>
                <!--
                As many sections as desired (at least 2) can be used
                -->
            </Section>
        </Sections>
        <Airfoils>
            <Airfoil>
                <!--
                Airfoil definition
                -->
            </Airfoil>
            <Airfoil>
                <!--
                As many airfoils as desired (at least 1) can be used
                -->
            </Airfoil>
        </Airfoils>
    <Wing>
        <!--
        As many wings as desired (at least 1) can be used
        -->
    </Wing>
</Aircraft>
\end{verbatim}

\noindent Note that all inputs in the geometry file are required unless
otherwise noted.

\subsubsection{Reference}

\begin{itemize}
	\item RefArea: Float. Description: The reference area used for computing
		force and moment coefficients.
	\item RefLength: Float. Description: The reference length used for computing
		the pitching moment coefficient.
	\item MomentCenX: Float. Description: The x-coordinate of the point about
		which the pitching moment is computed.
	\item MomentCenY: Float. Description: The y-coordinate of the point about
		which the pitching moment is computed.
	\item MomentCenZ: Float. Description: The z-coordinate of the point about
		which the pitching moment is computed.
\end{itemize}

\subsubsection{Wing}

\begin{itemize}
	\item Name (optional attribute): If included, this name will be used for
		per-wing output files. Otherwise, a name will be automatically generated
		based on a sequential numbering of the wings in the input file. To set
		this attribute, use the following syntax:
		\begin{verbatim}
		<Wing name="WingName">
		\end{verbatim}
	\item Paneling
	\begin{itemize}
		\item NChord: Integer. Description: Number of points along the chord
			on both the top and bottom surfaces. The total number of panels
			around the chord is $2\times\text{NChord} - 2$.
		\item NSpan: Integer. Description: Number of points along the span on
			the right half of the wing. The total number of panels along the
			span, including both the left and right side of the wing, is
			$2\times\text{NSpan} - 2$.
		\item{LESpaceRat}: Float. Description: Ratio of point spacing at the
			leading edge to uniform spacing. Uniform spacing is computed as the
			total arc length of the section divided by the number of panels
			around the chord.
		\item{TESpaceRat}: Float. Description: Same as LESpaceRat, but applies
			to the trailing edge.
		\item{RootSpaceRat}: Float. Description: Ratio of point spacing at the
			root relative to uniform spacing. Uniform spacing is computed as the
			``flat'' length of the wing (if dihedral were removed) divided by
			the number of panels along the span.
		\item{TipSpaceRat}: Float. Description: Same as RootSpaceRat, but
			applies to the tip.
	\end{itemize}
\end{itemize}

\begin{itemize}
	\item Section: Note that sections can be listed in any order; they are not
		required to be ordered from root to tip. There is a distinction between
		the sections that are listed in the XML file and the sections pertaining
		to the numerical method. The sections listed here define the shape of
		the wing, but the discretization process places NSpan sections along the
		span as defined by the RootSpaceRat and TipSpaceRat spacing ratios. To
		ensure the desired planform shape is preserved perfectly, the spacing
		algorithm is constrained to place a section at each of the spanwise
		positions defined here.
	\begin{itemize}
		\item Name (optional attribute): If name=``Root'' is set for this
			attribute, then the section will automatically be placed at Y=0, and
			Y does not need to be set for the section.
		\item XLE: Float. Description. The x position of the
			leading edge of the section, prior to twist.
		\item Y: Float. Required: Yes, except if name=``Root'' attribute is set.
			Description: The y position of the section.
		\item ZLE: Float. Description: The z position of the
			leading edge of the section, prior to twist.
		\item Chord: Float. Description: Chord length of the
			section.
		\item Twist: Float. Description: Twist of the section
			about the quarter-chord, in degrees. A positive twist is
			leading-edge-up.
	\end{itemize}
\end{itemize}

\begin{itemize}
	\item Airfoil: Note that airfoils can be listed in any order; they are not
		required to be ordered from root to tip. At least one airfoil is
		needed, but any number of airfoils can be placed along the span.
		LORAAX linearly interpolates airfoil coordinates to the discretized
		spanwise section locations.
	\begin{itemize}
		\item Name (optional attribute): If name=``Root'' or name=``Tip'' is set for
			this attribute, then the airfoil will automatically be placed at
			Y=0 or the tip location, and Y does not need to be set for this
			airfoil.
		\item Y: Float. Required: Yes, except if name=``Root'' or name=``Tip''
			attribute is set. Description: The y position of the airfoil.
		\item Source: String. Description: Specifies how the airfoil coordinates
			will be created. Either ``4 digit,'' ``5 digit,'' or ``File.'' 
		\item Camber: Float. Required: Only if Source is ``4 digit''. Description:
			Maximum camber of the mean camber line, as a percent, for a NACA 4 digit
			airfoil.
		\item XCamber: Float. Required: Only if Source is ``4 digit''. Description:
			Location of the max camber point, as x/c, for a NACA 4 digit airfoil.
		\item Thickness: Float. Required: Only if Source is ``4 digit''.
			Description: Maximum thickness, as a percent, for a NACA 4 digit airfoil.
		\item Designation: String. Required: Only if Source is ``5 digit''. This is
			the code defining a 5-digit NACA airfoil.
		\item Path: String. Required: Only if Source is ``File.'' This is the
			relative or absolute path to the airfoil coordinates file.
			Coordinates should be formatted in a single loop starting from the
			trailing edge, either of the top surface or bottom surface, wrapping
			around the leading edge and on to the trailing edge of the opposite
			surface. Coordinates should be arranged in two space-delimited
			columns representing
			x and y in 2D space. The first line can either be text (usually the
			name of the airfoil) or the first coordinate pair.
	\end{itemize}
\end{itemize}

\section{Running a Case}

After setting up the input files, a case can be run using the following
command:

\begin{verbatim}
loraax analysis_input_file.xml
\end{verbatim}

\noindent This assumes that the loraax executable can be found by the shell,
usually by placing its containing directory in the PATH environment variable. If
loraax is not in PATH, it will have to be referenced with an absolute or
relative path in the command above. The analysis input file must also be
referenced with an absolute or relative path if it is not in the current working
directory. Similarly, any files referenced within the analysis and geometry XML
files must be referenced by an absolute or relative path if they are not in the
current working directory.

When a case is run, information about each iteration is printed to the shell,
including the current operation that is being performed and, at the end of the
iteration, a summary of the lift, drag, and pitching moment coefficients.

\subsection{Solution Convergence}

The solution is considered to be converged when the change in lift coefficient
between interations is less than the specified StoppingTolerance. When the
solution converges, ``Solution is converged'' is printed and iterations end,
unless the iteration count is less than MinIters.

If the solution fails to converge, it may be necessary to increase MaxIters.
However, some cases may never converge, particularly if the case is viscous and
the angle of attack is too high. Xfoil, the boundary layer method employed by
LORAAX, can handle mild separation up to the onset of stall, but not beyond
that. When Xfoil fails to converge for a section, warnings are printed to the
shell and LORAAX attempts to approximate the boundary layer solution by
interpolating or extrapolating from nearby sections. Often, one or two sections
may not converge at some points during a simulation, even at fairly low angles
of attack. This situation does not significantly degrade the solution. However,
if a large percentage of the sections still fail to converge after 10 -- 20
solver iterations, it is likely that either stall occurring or XfoilRunOptions
$\rightarrow$ maxit may need to be increased. Be aware that the stall angle
of attack decreases as Reynolds number decreases, and jagged or excessively
coarse airfoil point distributions may degrade convergence. The discretization
method in LORAAX uses spline fitting to interpolate points on the airfoil
surface, but this method will not correct bumpy or jagged input
coordinate distributions. Coordinates can sometimes be smoothed manually
using the MDES $\rightarrow$ FILT
operation in Xfoil and saving the smoothed coordinates.

\subsection{Output Files}\label{sec:output}

When LORAAX runs, it creates the following three directories:

\begin{itemize}
	\item forcemoment
	\item sectional
	\item visualization
\end{itemize}

Additionally, if these directories already exist, a backup directory will be
created where the existing directories will be moved and renamed with a
timestamp.

\section{Analyzing / Postprocessing Results}

Postprocessing a case can be done in a number of ways, depending on what
information is desired. If only the final lift, drag, and pitching moment
coefficients are desired, the information printed to the shell can be used
directly (or redirected to a file). For more detailed analyses, the output files
in the directories from Section \ref{sec:output} should be used. A description
of the contents of each of these directories is given below.

\begin{itemize}
	\item forcemoment: Contains CSV-formatted files listing forces, moments, and
		coefficients for the aircraft at each iteration. Separate files are
		written for each wing and the total aircraft. Total aircraft
		coefficients are normalized by the reference information in the geometry
		input XML file, while coefficients for individual wings are normalized
		by that wing's planform area and mean aerodynamic chord. These files
		can be loaded and plotted in ParaView or by using a spreadsheet or
		script.
	\item sectional: Contains CSV-formatted files containing the lift
		coefficient, drag coefficient, and pitching moment coefficients per unit
		span, computed at each section. For viscous cases, the local Reynolds
		number is also written. One file is written for each wing at an interval
		controlled by the VisualizationFrequency input setting. These files can
		be loaded and plotted in ParaView or by using a spreadsheet or script.
	\item visualization: Contains VTK (legacy ASCII format) files to visualize
		the wing surface and wake. All relevant wing surface variables,
		including pressure, velocity, and boundary layer quantities, are
		included. One file each is written for surfaces and wakes at an interval
		controlled by the VisualizationFrequency input setting. These files can
		be loaded and visualized in ParaView or any other visualization package
		that supports the legacy VTK format.
\end{itemize}

\end{document}
